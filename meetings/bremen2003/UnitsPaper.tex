\documentclass[11pt]{openmathTN}
\def\action#1{\hfill\rlap{\bf #1}}
\usepackage{url}
\title{Units and Dimensions in OpenMath}
\author{James H.~Davenport}
\address{Department of Computer Science\\
University of Bath\\
Bath BA2 7AY\\
England\\
}
\author{William A.~Naylor}
\address{Victoria University of Wellington\\
New Zealand\\
}

\begin{document}
\begin{abstract}
We first review the current state of units and dimensions in OpenMath, and
compare it with the state of play in MathML
\cite{WorldWideWebConsortium2003}. We then discuss the fundamental unsolved
question: how does one write ``2 grammes'' in OpenMath? We present the
various solutions, and recommend one to the OpenMath community for debate
and decision. We then do the same for the follow-up question: how does one
write ``2 kilometres'' in OpenMath?
We conclude with a set of principles to guide the evolution of the
definition and use of units and dimensions in OpenMath.
\end{abstract}

\maketitle
\tableofcontents
\newpage

\section{Units in OpenMath: the current state}
The current\footnote{As at 1 July 2003.} state of units and dimensions in
OpenMath is given in the following Content Dictionaries:
\begin{enumerate}
\item {\tt dimension1.ocd} --- version 3 dated 2002-09-17;
\item {\tt units\_metric1.ocd} --- version 2 dated 2000-08-13;
\item {\tt units\_imperial1.ocd} --- version 2 dated 2000-08-13;
\end{enumerate}
and the corresponding STS \cite{Davenport2000c} files.
\par
There is no architecture document, but the following general principles
were followed, and can be reconstructed from the files.
\subsection{Dimensions}\label{areasect}
\begin{enumerate}
\item\label{area}The ordinary symbols from {\tt arith1.ocd} are used to compose
dimensions, e.g. area is defined with the following Formal Mathematical
Property:
\begin{verbatim}
<FMP><OMOBJ>
  <OMA>
    <OMS cd="relation1" name="eq"/>
    <OMS cd="dimensions1" name="area"/>
    <OMA>
      <OMS cd="arith1" name="times"/>
      <OMS cd="dimensions1" name="length"/>
      <OMS cd="dimensions1" name="length"/>
    </OMA>
  </OMA>
</OMOBJ></FMP>
\end{verbatim}
This seems compatible with the traditional usage in dimensional analysis.
One could argue that this is not the ``normal'' meaning of this {\tt
times}, but Occam's razor would operate against a new symbol. Though not
binding, we note that MathML \cite[section 3]{WorldWideWebConsortium2003}
appears to recommend the use of their standard {\tt times} content
symbol for multiplication of units\footnote{MathML seems to have no
particular concept of dimension, apart from a few words in \cite[section
6.1 and Appendix C]{WorldWideWebConsortium2003}.}.
\item\label{dimtype}The names of dimensions are used as types in the
OpenMath Small Type System \cite{Davenport2000c} for units, as in the
following signature for {\tt metre}.
\begin{verbatim}
<Signature name="metre" >
<OMOBJ>
  <OMS cd="dimensions1" name="length"/>
</OMOBJ>
</Signature>
\end{verbatim}
This is clearly sensible, and in line with the r\^ole of dimensions in
physics.
\end{enumerate}
\subsection{Metric Units}\label{metricsect}
\begin{enumerate}
\item\label{FMPunit}
There are no formal or commented mathematical properties in this file. This
is probably due to the absence of any agreed mechanism for attaching units
to quantities, or whether it would be legal OpenMath to write, for example,
the definition of square metre\footnote{Whether one should define such a
compound unit is a different matter. MathML (see item \ref{compoundMML} in
section \ref{MathMLsect}) would not, and indeed leaving it as ${\tt
metre}^2$ is simpler for renderers and reasoners. This question is asked as
\ref{compoundOM} in section \ref{OMqueries}.}, as in table
\ref{squaremetre}. The authors believe that this is legal OpenMath.
\begin{table}[ht]
\caption{Putative definition of square metre}\label{squaremetre}
One possible definition of a square metre to go into \verb+units_metric1+.
\begin{verbatim}
<FMP><OMOBJ>
  <OMA>
    <OMS cd="relation1" name="eq"/>
    <OMS cd="units_metric1" name="metre_sqrd"/>
    <OMA>
      <OMS cd="arith1" name="times"/>
      <OMS cd="units_metric1" name="metre"/>
      <OMS cd="units_metric1" name="metre"/>
    </OMA>
  </OMA>
</OMOBJ></FMP>
\end{verbatim}
\end{table}
\item The \verb+Newton_per_sqr_metre+ is defined, but not its ``proper''
name, the \verb+Pascal+. {\it Now fixed: \verb+Newton_per_sqr_metre+
retained for compatibility\footnote{The OpenMath standard \cite{OMSTD}
forbids removing symbols from CDs, but, since the {\tt units\_metric1} CD
is not yet official, this ban probably does not apply.}, but
deprecated\/}.
\item There are inconsistencies on the capitalisation of unit names. Those
that are directly people's names are capitalised (in accordance with the
principles in \cite{Davenport2000b}), but those that are merely based on them
(e.g. \verb+volt+ from Volta, or {\tt amp} from Amp\`ere) are not, even
though the abbreviation is in fact capitalised. This needs to be checked
with the appropriate standards.\action{JHD}
\end{enumerate}
\subsection{Imperial Units}
\begin{enumerate}
\item Despite the idea that FMPs could be used to do conversion, there are
none in this file. Probably this is due to the problem described in item
\ref{FMPunit} in the previous section.
\item Curiously, the {\tt mile} is defined, but not the {\tt foot}, which
is now\footnote{And has been since the time of Henry II, the first King of
England to demonstrate a knowledge of the Central Limit Theorem.} the
standard of length, being fixed at 0.3048m.
\end{enumerate}
\subsection{U.S. Units}
Nothing has so far been written on this subject apart from a note from JHD
saying that it would be the existence of this CD (read ``namespace'') that
distinguishes
\begin{verbatim}
<OMS name="pint" cd="units_imperial1"/>
\end{verbatim}
from
\begin{verbatim}
<OMS name="pint" cd="units_us1"/>
\end{verbatim}
We should note that this taxonomy is the other way up from that of MathML
--- see item \ref{variants} in section \ref{MathMLsect}.
\section{Units in MathML (Content)}\label{MathMLsect}
It is beyond the scope of this document to consider the presentation of
units in general, though the authors see no real problems with adapting to
the presentation view of units in \cite{WorldWideWebConsortium2003}.
\par
The main points relevant to units in content MathML in
\cite{WorldWideWebConsortium2003} are given below.
\begin{enumerate}
\item{}\label{timesunit}``In expressing a quantity with units, \dots, it is
recommended that the unit be the last child of of an \verb+apply+ element
which has the \verb+times+ element as its first child.'' \cite[section
3]{WorldWideWebConsortium2003}
\par
This would seem to imply that 3.4m would be represented as
\begin{verbatim}
<apply>
  <times/>
  <cn> 3.4 </cn>
  <csymbol> m </csymbol>
</apply>
\end{verbatim}
\item{}\label{times}``It would also be best if compound units are kept
separate as a nested apply at the end of a product.''  \cite[section
3]{WorldWideWebConsortium2003}
\par
This would seem to imply that 3.4m/sec would be represented as
\begin{verbatim}
<apply>
  <times/>
  <cn> 3.4 </cn>
  <apply>
    <divide/>
    <csymbol> m </csymbol>
    <csymbol> sec </csymbol>
  </apply>
</apply>
\end{verbatim}
The use of \verb+<divide/>+ is hypothetical here, but see the next point.
\item\label{compoundMML}``The \verb+csymbol+ element should be used to
identify units. It should not be used to encapsulate a compound units ({\it
sic}) as in the following two representations of the compound unit of speed
centimeters per second.
\begin{verbatim}
<csymbol>centimeters per second</csymbol> 
<csymbol>cm/s</csymbol> 
\end{verbatim}
Such compound units should be represented by explicit products or quotients
of simple units.'' \cite[section 5]{WorldWideWebConsortium2003} 
\par We note though, that in \cite[section
6.2]{WorldWideWebConsortium2003}, the authors use a \verb+csymbol+ to
denote ``watt'', so the intention seems to be to bar those compound units
that are comprised of multiple ``recognised'' units, not all those that can
be reduced to basic units. This means that the Content MathML expressions
for ``1Nm'' and ``1W'' would presumably be rather different, as in table
\ref{NmW}.
\begin{table}[ht]
\caption{Putative MathML for ``1Nm'' and ``1W''}\label{NmW}
\begin{verbatim}
<apply>                            <apply>
 <times>                            <times>
 <cn> 1 </cn>                       <cn> 1 </cn>
 <apply>                            <csymbol definitionURL="...">
  <times>                            W</csymbol>
  <csymbol definitionURL="...">    </apply>
    N</csymbol>
  <csymbol definitionURL="...">
    m</csymbol>
 </apply>
</apply>
\end{verbatim}
\end{table}
\item\label{MathMLprefix}MathML seems to regard, say, {\tt centimetre}
({\tt cm}) as a separate unit, not a derivation from {\tt metre} ({\tt m}),
though this may only be in the context of the CGS system \cite[section
6.5.1]{WorldWideWebConsortium2003}. At other times it uses the fragment
identifier on URLs, as in
\begin{verbatim}
<csymbol definitionURL='http.../units/pascal#k'>kPa</csymbol>
\end{verbatim}
({\it loc.cit.\/}), though this is wrapped in a complex {\tt semantics}
encoding which also includes (our re-formatting)
\begin{verbatim}
<annotation-xml encoding='MathML'
                definitionURL='http.../SI-conversion-factor'>
<cn type='integer'>1000</cn>
</annotation-xml>
\end{verbatim}
This seems to imply that the meaning of {\tt k} is local to this piece of
MathML, which seems obscure.
\item\label{variants}MathML takes the approach of naming the unit, and then
the variant, as in \cite[section 5.3.4]{WorldWideWebConsortium2003}
\url{http://.../units/mile/gb-sct} for the Scottish mile, or
\url{http://.../units/mile/survey/us}. In the latter case, the authors would
suggest that the OpenMath name should be
\begin{verbatim}
<OMS name="survey_mile" cd="units_us1"/>
\end{verbatim}
thus placing the system higher up the hierarchy than the name. The first
example is more problematic, but could be represented by
\begin{verbatim}
<OMS name="scottish_mile" cd="units_british_historic"/>
\end{verbatim}
\end{enumerate}
\section{The Fundamental Question --- Attribution}\label{attributed}
How should we signify that a quantity (ultimately numeric, but which may be
a symbolic expression) is attributed with a unit? More specifically, since
\begin{verbatim}
<OMI> 2 </OMI>
\end{verbatim}
is the OpenMath for the number 2, and
\begin{verbatim}
<OMS name="gramme" cd="units_metric1"/>
\end{verbatim}
is the OpenMath for a gramme, what is the OpenMath for "2 grammes"? There
seem to be three options.
\begin{enumerate}
\item\label{timesoption}Use the {\tt times} from {\tt arith1}. Then
  the encoding for 2 grammes would be:
\begin{verbatim}
<OMA>
  <OMS name="times" cd="arith1"/>
  <OMI> 2 </OMI>
  <OMS name="gramme" cd="units_metric1"/>
</OMA>
\end{verbatim}
\begin{description}
\item[pro]Aids MathML compatibility and conversion --- see item
\ref{timesunit} in section \ref{MathMLsect}.
\item[pro]Requires no extra symbols, and indeed means that a system that
does not fully understand units can at least manipulate them as symbolic
variables. \cite{Khanin2001}?
\item[con]Somewhat of a perversion of this symbol, since what is going on
is not multiplication in any homogeneous sense. Since this {\tt times} is
declared to be {\tt nassoc} \cite{Davenport2000c}, this means that the
result of multiplying two Newtons by three metres is
\begin{verbatim}
<OMA>
  <OMS name="times" cd="arith1"/>
  <OMI> 2 <\OMI>
  <OMS name="Newton" cd="units_metric1"/>
  <OMI> 3 <\OMI>
  <OMS name="metre" cd="units_metric1"/>
</OMA>
\end{verbatim}
whereas the OpenMath equivalent of the MathML way of representing this (see
table \ref{NmW}) would be as follows.
\begin{verbatim}
<OMA>
  <OMS name="times" cd="arith1"/>
  <OMI> 2 <\OMI>
  <OMI> 3 <\OMI>
  <OMA>
    <OMS name="times" cd="arith1"/>
    <OMS name="Newton" cd="units_metric1"/>
    <OMS name="metre" cd="units_metric1"/>
  </OMA>
</OMA>
\end{verbatim}
\end{description}
\item Using a different symbol, with different semantics, e.g. possibly
only binary. It is not totally clear what the signature would be: probably
$$
\verb+NumericalValue+\times\verb+Dimension+\rightarrow\verb+DimensionedValue+.
$$
Possibly the best choice might be
\begin{verbatim}
<OMS name="times" cd="units_ops"/>
\end{verbatim}
\begin{description}
\item[pro]Unambiguous, and systems that can't understand units would not
support this operator.
\item[con]Occam's razor, and the greater complexity that a new symbol,
which would need to be widely understood, would bring to OpenMath.
\end{description}
\item\label{unitsattributes}Use OpenMath attributes. Then the encoding for
the problem that started this section would be as follows.
\begin{verbatim}
<OMATTR>
  <OMATP>
    <OMS name="quantity_with_units" cd="units_ops"/>
    <OMS name="gramme" cd="units_metric1"/>
  </OMATP>
  <OMI> 2 </OMI>
<\OMATTR>
\end{verbatim}
\begin{description}
\item[pro]Attributes are already built into OpenMath.
\item[con]Attributes can be silently discarded by applications, which seems
dangerous for something as vital, and mission-crashing, as units.
\end{description}
\end{enumerate}
Without wishing to prejudge the debate in the OpenMath community, the rest
of this document, and the associated CDs, use option \ref{timesoption}
above.
\section{Unit Prefixing}\label{prefixing}
The SI system\footnote{And others. In particular there is the IEC scheme
for binary prefixes \cite[Appendix B.2]{WorldWideWebConsortium2003}.} has a
uniform scheme for prefixing units to create larger or smaller one,
generally in multiples of $10^3$. How can we apply this scheme uniformly in
OpenMath?
\par
As in the previous section, there are various possibilities.
\begin{enumerate}
\item Treat them as separate units, as MathML sometimes seems to do: see
item \ref{MathMLprefix} in section \ref{MathMLsect}.
\begin{description}
\item[pro]No new mechanisms involved.
\item[con]Very verbose: each unit would also acquire 20 variants.
\item[con]Problems of consistency.
\end{description}
\item As MathML (see item \ref{MathMLprefix} in section \ref{MathMLsect},
or \cite[section 5.3.5]{WorldWideWebConsortium2003}) does, use the XML
`prefix' notation: \verb+#+. Hence in MathML, the kiloPascal {\tt kPa} has
a {\tt definitionURL} of \verb+http://.../units/pascal#k+. Of course,
strictly speaking this is now a URI reference, rather than a URL, but this
seems to be becoming common in MathML. More seriously, OpenMath already
uses the prefix notation to distinguish individual symbols in a CD. Hence
this is not a viable option for OpenMath.
\begin{description}
\item[pro]None that I can see for OpenMath, as opposed to MathML.
\item[con]Unusable\footnote{This has been checked with David Carlisle.} in
OpenMath.
\item[con]Contrary to the OpenMath philosophy \cite{OMSTD} of semantic
extensibility through CDs, requires every application to understand this
\verb+#+ syntax.
\item[con]The $10^{-6}$ prefix, $\mu$, has to be replaced by \verb+u+,
which means that faithful rendering needs an extra rule to translate
\verb+u+ into $\mu$.
\item[con]The connection between the syntax \verb+#k+ and the semantics of
$\times10^3$ seems to depend on the MathML-generator generating the correct
combination of \verb+definitionURL+ and \verb+annotation-xml+ attri\-butes.
\end{description}
\item Use of \verb+<OMS name="times" cd="arith1"/>+, as was suggested in the
previous section for attributing units.
\begin{description}
\item[pro]No extra syntax or semantics required, so that a
kiloPascal would be as follows.
\begin{verbatim}
<OMA>
  <OMS name="times" cd="arith1"/>
  <OMS name="kilo" cd="units_siprefix1"/>
  <OMS name="Pascal" cd="units_metric1"/>
</OMA>
\end{verbatim}
\item[con]A fairly simplistic simplifier, which knew about \verb+<OMS name="times"+
\verb+cd="arith1"/>+, might multiply kW by km to get
\begin{verbatim}
<OMA>
  <OMS name="times" cd="arith1"/>
  <OMS name="metre" cd="units_metric1"/>
  <OMS name="Watt" cd="units_metric1"/>
  <OMA>
    <OMS name="power" cd="arith1"/>
    <OMS name="kilo" cd="units_siprefix1"/>
    <OMI> 2 </OMI>
  </OMA>
</OMA>
\end{verbatim}
\item[con]There is no guarantee that prefixes will in fact be prefixes, and
certainly no rule to enforce this.
\end{description}
\item\label{prefixop}A special operator, such as
\verb+<OMS name="prefix" cd="units_ops1"/>+,
whose signature would be as given in table \ref{prefixsig}.
\begin{table}[ht]
\caption{Proposed signature of the ``prefix'' operation}\label{prefixsig}
\begin{verbatim}
<Signature name="prefix" >
<OMOBJ>
  <OMA>
    <OMS name="mapsto" cd="sts"/>
    <OMS cd="units_sts" name="unit_prefix"/>
    <OMV name="dimension"/>
    <OMV name="dimension"/>
  </OMA>
</OMOBJ>
</Signature>
\end{verbatim}
\end{table}
We recall that, by the rules of the Small Type System
\cite{Davenport2000c}, this means that the first argument must have type
\verb+unit_prefix+ (i.e. must be a prefix), the second argument can be anything
(the human reads ``something of an unknown dimension'') but must return
something of the {\it same\/} species (the human being reads ``same
dimension''). For example the prefixed unit {\em millisecond} could be
specified by the markup:
\begin{verbatim}
<OMA>
  <OMS name="prefix" cd="units_ops1"/>
  <OMS name="milli" cd="units_siprefix1"/>
  <OMS name="second" cd="units_time1"/>
</OMA>
\end{verbatim}

\begin{description}
\item[pro]The order of the arguments ensures that, at least locally,
prefixes are genuinely prefixes.
\item[pro]Prefixes do not complicate any type-checking and consistency
mechanism, as a result of that signature.
\item[con]An extra operator, and indeed an extra CD \verb+units_sts+ to
hold the necessary type.
\end{description}
\item Attributes, as discussed in solution \ref{unitsattributes} in section
\ref{attributed}.
\begin{description}
\item[pro]Attributes are already built into OpenMath.
\item[con]Attributes can be silently discarded by applications, which seems
even more dangerous than discarding units as a whole: what application
would notice that a "Mega" had silently vanished?
\end{description}
\end{enumerate}
Without prejudicing the conclusions of the OpenMath community, we will go
with option \ref{prefixop} in this document.
\section{Other Points for OpenMath}\label{OMqueries}
\subsection{Nomenclature}
Should not {\tt units\_metric1} be renamed {\tt units\_si1}, thus allowing
for CDs such as \verb+units_cgs1+, as well as for a
\verb+units_metricmisc1+, which might contain the (metric) horsepower, the
hectare\footnote{Unless we introduce the {\tt are} (little used in its own
right) as an alternative for ${\tt metres}^2$, and rely on prefixing (see
section \ref{prefixing}) to generate the hectare for us. However, might it
not generate the ``hectoare'' instead?}, and so on.
\subsection{Formal Mathematical Properties}
The ``defining mathematical property'', as discussed in Pisa on
28--29.9.2002 should be resurrected\footnote{\label{MKnote}With help from
MK, who said in Pisa that he had a solution.}, and applied to ensure that
all non-fundamental dimensions were defined, possibly indirectly\footnote{Note
that, in the Pisa proposal, the author(s) of defining mathematical
properties must ensure that the system is not circular.}, in terms of
fundamental ones. For example, the formal mathematical property in the
definition of {\tt area} discussed in item \ref{area} of section
\ref{areasect} should be such a defining mathematical
property.\action{JHD$\strut^{\ref{MKnote}}$}
\subsection{Particular Dimensions}
\subsubsection{The STS of dimensions}
At the moment, all dimensions have an STS entry as follows.
\begin{verbatim}
<OMV name="PhysicalDimension"/>
\end{verbatim}
Might it not make more sense to make this an {\tt OMS}, or possibly two
{\tt OMS}s --- one for fundamental\footnote{In theory, of course, the
choice of fundamental dimensions is arbitrary. However, convention has
fixed a set, and this choice would probably be less arbitrary than some
choices of branch cuts which OpenMath has already made
\cite{Corlessetal2000}.} dimensions and one for derived dimensions. This
distinction might help reasoners about units to know when no further
reduction was possible.
\subsubsection{The Electrical Dimensions}
These have no formal mathematical properties, and the commented ones are
circular. There needs to be a proper investigation of the state of the SI
system \cite{IEEEASTM1997} in this area, and the correct base
dimension\footnote{\cite[section 6.1]{WorldWideWebConsortium2003} implies
that this is current.} chosen and the rest reduced to it.\action{JHD}
\subsubsection{Concentration}
The current CMP defines it as $\frac{{\tt mass}}{{\tt length}^3}$, which
would mean that {\tt mole} would not be a unit of concentration.
Concentration could be regarded as `quantity/volume', but this poses the
question ``what is {`quantity'}''. This needs investigating.\action{JHD?}
\subsection{Compound Units}\label{compoundOM}
By ``compound unit'', we mean a unit that does not have a specific name in
\cite{IEEEASTM1997}, but is built up from other units. 
Should OpenMath, as it currently does, define compound units such as
\verb+metres_+\allowbreak\verb+sqrd+, or leave them as, for example, ${\tt
metres}^2$, as MathML does (see item \ref{compoundMML} in section
\ref{MathMLsect})? The authors are inclined to vote against compound units,
on the ground that they add little, and complicate life for renderers.
\subsection{Particular Units}
\subsubsection{{\tt units\_time1}}
Various other kinds of year, such as the sidereal year, and month, such as
the lunar month, need to be defined.\action{JHD}
%\subsection{Particular Prefixes}
%So far only {\tt kilo} and {\tt milli} among the decimal prefixes have been
%defined. Once the larger questions have been answered, the other SI
%prefixes \cite[Appendix B.1]{WorldWideWebConsortium2003} need to be
%defined. Also, the binary prefixes \cite[Appendix
%B.2]{WorldWideWebConsortium2003} need to be defined --- a start has been
%made with {\tt kibi} ($=2^{10}$).
\subsection{Currency}
No thought has yet been given in OpenMath to currency, either with
fixed rates (``100 cents = 1 dollar'')\footnote{Or the more unusual, such
as ``3 English marks = 2 pounds Sterling''. One might be tempted to class
``20 guineas = 21 pounds Sterling'' in this category, but in fact this is
only true since Sir Isaac Newton was Warden of the Royal Mint: prior to
that, the guinea floated against the pound Sterling.}, currently
fixed rates (``6.55957 French Francs = 1 Euro'') or variable rates. The first
seems to fit within the current paradigm, but it would probably be unwise
to press ahead with this without an eye to the broader picture.
\section{Questions for MathML}\label{MathMLfuture}
\begin{enumerate}
\item\cite[section 3]{WorldWideWebConsortium2003} (see item \ref{times} in
section \ref{MathMLsect}) should perhaps explicitly say that the first
child of the {\tt apply} defining the compound unit should be one of {\tt
times}, {\tt divide} (as in ``metres per second'') or {\tt power}.
\item Does MathML (Content) really wish to continue with defining ``m''
(metre) and ``cm'' (centimetre) as different units. See OpenMath's view in
section \ref{prefixing}.
\end{enumerate}
\section{Conclusion}
The following principles are used in constructing the existing dimensions
and units CDs, and should be followed in any additions to them.
\begin{enumerate}
\item The ordinary symbols from {\tt arith1.ocd} are used to compose
dimensions, as in point \ref{area} of section \ref{areasect}.
\item The names of dimensions are used as types in the OpenMath Small Type
System \cite{Davenport2000c} for units, as in point \ref{dimtype} of
section \ref{areasect}.
\item It is legal to declare that two dimensions, or dimensional
expressions, are equal directly, as in item \ref{area} of section
\ref{areasect}.
\item It is legal to declare that two units, or unit expressions, are equal
directly, as in item \ref{FMPunit} of section \ref{metricsect}. Of course,
if the equivalence is not one-for-one, then one needs to use the mechanism
of section \ref{attributed} to say that, say, one acre is 4840 square
yards.
\item The conclusion of section \ref{attributed}.
\item The conclusion of section \ref{prefixing}.
\item The conclusion of section \ref{compoundOM}.
\end{enumerate}
\begin{thebibliography}{9}
\bibitem{OMSTD}
{\sc Caprotti, O., Carlisle, D.P. and Cohen, A.M.}
\newblock {\em The OpenMath Standard}.
\newblock \url{http://www.openmath.org/std}

\bibitem{Corlessetal2000}
{\sc Corless, R.M., Davenport, J.H., Jeffrey, D.J. and Watt, S.M.}
\newblock {According to Abramowitz and Stegun}.
\newblock {\em ACM SIGSAM Bulletin 30(2)} 2000, 58--65.

\bibitem{Davenport2000b}
{\sc Davenport, J.H.}
\newblock {On Writing OpenMath Content Dictionaries}.
\newblock {\em ACM SIGSAM Bulletin 30(2)} 2000, 12--15.

\bibitem{Davenport2000c}
{\sc Davenport, J.H.}
\newblock {A Small OpenMath Type System}.
\newblock {\em ACM SIGSAM Bulletin 30(2)} 2000, 16--21.

\bibitem{IEEEASTM1997}
{\sc IEEE/ASTM}
\newblock {SI 10-1997 {\em Standard for the Use of the International System
of Units (SI): The Modern Metric System}}.
\newblock IEEE Inc., New York, 1997.

\bibitem{Khanin2001}
{\sc Khanin, R.}
\newblock {Dimensional Analysis in Computer Algebra}.
\newblock In {\em Proceedings ISSAC 2001\/} (2001), B.~Mourrain, Ed., ACM,
N.~York, pp. 201--208.

\bibitem{WorldWideWebConsortium2003}
{\sc World-Wide Web Consortium (ed. D.W. Harder and S. Devitt)},
\newblock {\em Units in MathML}.
\newblock {W3C Note, 1 July 2003}.
\url{http://www.w3.org/Math/Documents/Notes/units.xml}.
\end{thebibliography}
\appendix
\section{Changes made to the CDs}
This appendix documents the changes made to the Content Dictionaries by JHD
in the process of writing this document.
\subsection{{\tt dimensions1.ocd}}
\begin{enumerate}
\item FMP added for {\tt speed}, matching the CMP. This was not done for
{\tt velocity}, since it's not clear how to specify this formally.
Clarified, in the ``Description'' field, the difference between speed and
velocity. An FMP was added to {\tt velocity} stating that ``speed is the 
vector norm of velocity''.
\item In the CMP for {\tt acceleration}, ``metres'' was changed to
''length''. An FMP was added, based on that for {\tt speed}.
\item FMP added for {\tt force}, as
\begin{verbatim}
<FMP><OMOBJ>
  <OMA>
    <OMS cd="relation1" name="eq"/>
    <OMS cd="dimensions1" name="force"/>
    <OMA>
      <OMS cd="arith1" name="times"/>
      <OMS cd="dimensions1" name="mass"/>
      <OMS cd="dimensions1" name="acceleration"/>
    </OMA>
  </OMA>
</OMOBJ></FMP>
\end{verbatim}
In general, dimensions were related to their logical predecessors, rather
than to the base dimensions.
\item FMPs added for {\tt pressure}, {\tt density}\footnote{Or should this
be specific gravity?}, {\tt energy} (along with matching CMP) and
\verb+momentum+.
\item Added the dimension \verb+power+, with an FMP and corresponding STS.
\end{enumerate}
\subsection{{\tt units\_metric1.ocd}}
\begin{enumerate}
\item Enhanced opening comments to point out that it is SI units that are
defined in this file, not all metric units. Similarly, changed ``metric''
to ``SI'' where appropriate
\item Said that the {\tt second} here was the same as the {\tt second} in
\verb+units_time1+, the general time CD.
\item Enhanced description for {\tt gramme} to point out that, although
kilogramme is the SI unit of mass, we take {\tt gramme}, otherwise the
gramme would be the milli-kilogramme! Stated that it was a basic unit of
SI.
\item Gave \verb+metre_sqrd+ and \verb+litre+ FMPs in terms of {\tt metre}.
\item Is the litre SI? Yes, it is, but \cite[Section
5.3.3]{WorldWideWebConsortium2003}, it changed definition (by about
0.0028\%) in 1964. Therefore added, with a CMP but without a FMP,
\verb+litre_pre1964+\footnote{\cite[Section
5.3.3]{WorldWideWebConsortium2003} calls this {\tt liter/1901}, which is
indeed the date when the previous definition was adopted, but seems less
helpful than saying when it ceased to be the definition.} and suitable
clarifying text.
\item Added \verb+Joule+ and \verb+Watt+, with the corresponding STS.
\end{enumerate}
\subsection{{\tt units\_metric1.sts}}
\begin{enumerate}
\item Changed the signature for {\tt pint} ({\it sic}) to the
obviously-intended {\tt litre}.
\item Changed the STS for \verb+metres_per_second+ from {\tt velocity} to
{\tt speed}, to conform with the CMP in the {\tt.ocd} file.
\item Added \verb+Pascal+, defined the existing \verb+Newton_per_sqr_metre+
to be equivalent to it (as an FMP), but deprecated in OpenMath.
\end{enumerate}
\subsection{{\tt units\_imperial1.ocd}}
\begin{enumerate}
\item Added the {\tt foot}, since this is the base unit, and defined
(currently in a CMP and an FMP) as 0.3048 metres. Gave the mile a CMP and
an FMP in terms of the foot. Also added the STS.
\item Added comments to the {\tt mile} to point out that it was the land,
or statute, mile.
\item Added {\tt yard}, with a CMP and FMP in terms of the {\tt foot}, and
associated STS.
\item Gave {\tt acre} a CMP and FMP in terms of square yards.
\item Added comment to {\tt pint} to cross-reference to {\tt pint} in
\verb+units_us1+ (when written).
\item Added CMP and FMP to \verb+miles_per_hr+ and
\verb+miles_per_hr_sqrd+. 
\end{enumerate}
\subsection{{\tt units\_time1.ocd}}\label{timesect}
A new CD (and corresponding STS file), since time is common to, at least,
SI and Imperial systems of units, and probably also the US one, but this
needs checking.\action{JHD}
\begin{enumerate}
\item \verb+second+ is defined as the fundamental unit.
\item \verb+minute+, \verb+hour+, \verb+day+ and \verb+week+ are derived in
turn. The definition of \verb+day+ explicitly states that it does not take
account of leap seconds.
\item {\tt month} and {\tt year} are more complicated. There are a variety
of definitions, and the calendar definitions are not constant. We therefore
explicitly define \verb+calendar_month+ and \verb+calendar_year+, and will
in the future need to define other variants. \verb+calendar_month+ has the
following FMP, and \verb+calendar_year+ an equivalent one, as well as one
saying that a \verb+calendar_year+ is precisely\footnote{One could argue
about the appropriateness of this. After all, 12 Februarys are not a
calendar year. But there seems no way of expressing ``12 consecutive
months'', and the proposition is valid in that one year can always be
converted into twelve months.} 12 \verb+calendar_month+s.
\begin{verbatim}
<FMP><OMOBJ>
<OMA>
  <OMS name="in" cd="set1"/>
  <OMA>
    <OMS name="divide" cd="arith1"/>
    <OMS name="calendar_month" cd="units_time1"/>
    <OMS name="day" cd="units_time1"/>
  </OMA>
  <OMA>
    <OMS cd="interval1" name="integer_interval"/>
    <OMI> 28 </OMI>
    <OMI> 31 </OMI>
  </OMA>
</OMA>
</OMOBJ></FMP>
\end{verbatim}
\end{enumerate}
\subsection{{\tt units\_siprefix1.ocd}}
This contains all the SI prefixes. Note
that the powers of 10 involved are explicit, rather than being a series of
zeroes, or, worse, an approximate floating point number.
\subsection{{\tt units\_binaryprefix1.ocd}}
This contains all the IEC binary prefixes. Note
that the powers of 2 involved are explicit, rather than being a series of
digits, prone to error.
\subsection{{\tt units\_ops.ocd}}
So far, this only contains the {\tt prefix} operator, as defined in item
\ref{prefixop} in section \ref{prefixing}.
\subsection{{\tt units\_sts.ocd}}
So far, this only contains the type {\tt unit\_prefix}, the type of all unit
prefixes, and which is used in the STS for
\verb+<OMS name="prefix" cd="units_ops"/>+. The STS for {\tt unit\_prefix}
describes it as being of type
\begin{verbatim}
  <OMS name="Object" cd="sts"/>
\end{verbatim}
\end{document}
