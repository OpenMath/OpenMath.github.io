\def\arctanD{\underbrace{\mathop{\rm arctan}\nolimits{}}_{\mathop{\rm Derive}\nolimits}}
\def\arctanM{\underbrace{\mathop{\rm arctan}\nolimits{}}_{\mathop{\rm Maple}\nolimits}}
\def\Ei{\mathop{\rm Ei}\nolimits}
\def\ber{\mathop{\rm ber}\nolimits}
\def\bei{\mathop{\rm bei}\nolimits}
\def\J{\mathop{\rm J}\nolimits}
\def\arccot{\mathop{\rm arccot}\nolimits}
\def\eq{\mathrel{{\tt eq}}}
\def\equal{\mathrel{\hbox{\tt =}}}
\def\interval#1#2{[#1,#2]}
\def\ttinterval#1#2{\hbox{\tt[}#1,#2\hbox{\tt]}}
\def\GF{{\rm GF}}
\def\factor{{\rm factor}}
\def\trdeg{{\rm tr.deg.}}
\def\atan{{\rm atan}}
\def\deg{{\rm deg}}
\def\signum{{\rm signum}}
\def\dx{\,{\rm d}x}
\def\Z{{\bf Z}}
\def\C{{\bf C}}
\def\Q{{\bf Q}}
\def\R{{\bf R}}
\def\RR{\R\rightarrow\R}
\documentclass[11pt,a4paper]{artikel3}
\usepackage{amsmath}
\newtheorem{Proposition}{Proposition}
\newtheorem{Definition}{Definition}
\newtheorem{Theorem}{Theorem}
\begin{document}
\title{Special Functions in OpenMath}
\author{James H. Davenport\\
Department of Computer Science,\\
University of Bath, Bath BA2 7AY, England\\
{\tt   J.H.Davenport@bath.ac.uk}\\
partially supported by the EU's OpenMath Thematic Network}
\date{}
\maketitle
\par\noindent
With thanks to St\'ephane Dalmas and the INRIA/UNSA team; Bruce Miller at
NIST, and Bill Naylor (Bath and U.W.O.).
\par\noindent
As usual, all errors are mine.
%\vskip 1.53in
\section{Preliminaries}
The following principles were endorsed at the OpenMath Meeting at ISSAC
2001 and at the OpenMath Network Meeting in Berlin (August 2001).
\begin{itemize}
\item Many CDs, rather than a single ``special functions CD''.
\item Short names, such as $\J$, similar to mathematical notation.
\item Use currying where appropriate, so that we can talk about $\J_\nu$
rather than having to talk about $\lambda z:\J(\nu,z)$.
\item As in the work on elementary functions, to stick as closely as
possible to Abramovitz and Stegun, and to be as precise as possible.
\end{itemize}
%\vskip 1.03in
\subsection{Difficulties with these}
\begin{itemize}
\item Just how many CDs? A CD for Laguerre, with one symbol in, or a CD for
all families of orthogonal polynomials, or what?
\item[]{\bf I have probably used too many CDs.}
\item Just what names are appropriate?
\item Where is currying appropriate?
\item[]{\bf Please tell me where you think I've got this wrong.}
\item How much should we follow A+S: symbols defined in footnotes or as
alternates? Do we need erfc? Every last {\it positive\/} wrinkle??
\item A+S is sometimes not helpful: see chapters 9 (Bessel for integer
order; but formulae valid for all orders) vs 10 (Bessel of fractional
order, meaning $n+\frac12$).
\item A+S has bugs: 6.1.23: $\ln\Gamma(\overline z)=\overline{\ln\Gamma(z)}$
is false: $z=-2.5$.
\end{itemize}
%\vskip 1.03in
\subsection{Difficulties with OpenMath}
\begin{itemize}
\item Functions defined by ordinary differential equations:
\item[]{\it need a way of specifying the differential equations and
boundary/initial conditions\/}
--- {\tt odesoln1.ocd}.
\item Functions defined by contour integrals:
\item[]{\it we need a way of specifying contours and the restrictions on them\/}
--- {\tt contour1.ocd}.
\item No direct way of specifying the $n$-th total derivative: this is
probably a bug even in MathML conformity.
\item Need to specify a restricted range, e.g. only $g_2$ and $g_3$ exist as
Weierstrass invariants:
\item[]{\it extend the use of STS\/} (but not the syntax).
%\end{itemize}
%\vskip 0.3in
%\subsection{Difficulties with OpenMath --- II}
%\begin{itemize}
\item Need to specify Cauchy principal value of integrals: an attribute or
a new symbol?
\item[]{\it New symbol in\/} {\it calculus2.ocd}.
\item The 8+3 rule proves restrictive: I want to put the Weierstrass
symbols in\verb+weierstrass.ocd+, not \verb+weierstr.ocd+. In fact, this
rule is not followed already --- see \verb+relation1.ocd+ for example.
\item Hard to express statements about uniqueness of solutions, or linear
independence of solutions.
\end{itemize}
%\vskip 1.73in
\subsection{{\tt odesoln1.ocd}}
\begin{verbatim}
<OMBIND>
  <OMS name="ODEsolution" cd="odesoln1"/>
  <OMBVAR>
    <OMV name="f"/>
    <OMV name="x"/>
  </OMBVAR>
\end{verbatim}
Expression in $f$ and $x$ defining LHS of ODE (=0).
\begin{verbatim}
  <OMA>
    <OMS name="list" cd="list1"/>
\end{verbatim}
List of expressions in $f$ representing initial/{\it boundary\/} conditions.
\begin{verbatim}
  </OMA>
</OMBIND>
\end{verbatim}
%\vskip 0.7in
\subsection{{\tt contour1.ocd}}
The symbol \verb+path_description+ describes a contour. possibly not
uniquely.
\begin{verbatim}
A contour from t to infinity, not crossing the
negative real axis.
<OMOBJ>
  <OMBIND>
    <OMS name="path_description" cd="contour1"/>
    <OMBVAR>
      <OMVAR name="P"/>
    </OMBVAR>
    <OMV name="t"/>
    <OMS name="infinity" cd="nums1"/>
    <OMA>
      <OMS name="not" cd="logic1"/>
      <OMBIND>
        <OMS name="exists" cd="quant1"/>
        <OMBVAR>
          <OMV name="x"/>
        </OMBVAR>
        <OMA>
          <OMS name="and" cd="logic1"/>
          <OMA>
            <OMS name="gt" cd="relation1"/>
            <OMV name="x"/>
            <OMS name="zero" cd="alg1"/>
          </OMA>
          <OMA>
            <OMS name="le" cd="relation1"/>
            <OMA>
              <OMS name="real" cd="complex1"/>
              <OMA>
                <OMV name="P"/>
                <OMV name="x"/>
              </OMA>
            </OMA>
            <OMS name="zero" cd="alg1"/>
          </OMA>
          <OMA>
            <OMS name="eq" cd="relation1"/>
            <OMA>
              <OMS name="imaginary" cd="complex1"/>
              <OMA>
                <OMV name="P"/>
                <OMV name="x"/>
              </OMA>
            </OMA>
            <OMS name="zero" cd="alg1"/>
          </OMA>
        </OMA>
      </OMBIND>
    </OMA>
  </OMBIND>
</OMOBJ>
\end{verbatim}
%\vskip 0.4in
\section{The Real Problem: Analytic Continuation}
$$\Ei(x)=-\int_{-x}^\infty\frac{e^{-t}}t\,{\rm d}t:\qquad x>0.$$
But, $\Ei$ is defined by analytic continuation for all $z$, continuous
modulo a branch cut along the negative real axis.
\par\noindent
How do we define this?
\begin{itemize}
\item Define a domain for $\Ei$: $\C\setminus\{0\}$.
\item[]But how do we define the branch cut?
\item Define a domain of continuity, i.e.
$\C\setminus\{\Re(z)<0;\Im(z)=0\}$.
\item[]Then how do we say which side the branch cut adheres to?
\end{itemize}
Suggestions welcome.
\section{Bessel Functions}
\begin{itemize}
\item $\J_\nu$ and $\J_{-\nu}$ are two independent solutions to Bessel's
equation.
\item Except when $\nu$ is an integer, when 
$$
Y_\nu(z)=\lim_{\mu\rightarrow\nu}\frac{\J_\mu(z)\cos(\mu\pi)-\J_{-\mu}(z)}{\sin(\mu\pi)}
$$
is the second solution.
\item Unfortunately, the initial conditions at $z=0$ are not always
easy to state. 
\item $\ber_\nu(x)+i\bei_\nu(x)=\J_\nu(xe^{3\pi i/4})$ --- do we need this,
or the very many other variants?
\item Need a domain expert (DLMF?)!
\end{itemize}
\end{document}
